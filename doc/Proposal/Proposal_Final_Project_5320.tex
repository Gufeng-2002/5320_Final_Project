\documentclass{article}
\usepackage{graphicx}
\usepackage{hyperref}
\usepackage{amsmath}
\usepackage{booktabs}
\usepackage{longtable}
\usepackage{geometry}
[
    a4paper,
    left=1in,
    right=1in,
    top=1in,
    bottom=1in
]

\title{Proposal for the Final Project STAT 5320}
\author{Anti Li(T00751339), Feng Gu(T00751197), Yuzhuo Ye(T00751492)}
\date{\today}

\begin{document}

\maketitle
\section{Team members}
Our group has three members:

\textbf{Anti Li(T00751339), Feng Gu(T00751197), Yuzhuo Ye(T00751492)}.

\section{Meeting and discussion}
We discussed the project with one meeting and one discussion. The meeting was on Mar 12 afternoon,
in person. The discussion was held online.

\textbf{Discussion(online):}
\begin{itemize}
    \item Time: Mar 9, around 18:00
    \item Content:
        Discussed the major task type in our project, and talked about the skills might be used in the project.
        Decided the formal meeting time and the possible data set we might choose.
\end{itemize}

\textbf{Meeting(in person):}
\begin{itemize}
    \item Time: Mar 12, from 14:00 to 15:40
    \item Content:
        Discussed the general idea, steps and the exercises that can be finished in the project. Decided the data set we are going to use.
        Divided the work into different parts and roughly assigned the tasks to each member.
\end{itemize}

\section{Data set to be used}
The data set we are going to use is the \textbf{Energy Appliances} data set. 
Its shape is \textbf{(19735, 29)} and the data is about the energy consumption of appliances in a low energy building.
The data set is collected in a house with a total of 10 electrical appliances. 
It is noticeable that the data was collected every 10 minutes for about 4.5 months,
which means there are some time series features in it. 
However, we would not focus on the time series analysis
that is beyond the scope of this project and thereby we would make some assumptions to simplify our analysis.

Table \ref{tab:energy_appliances} shows the data variables in the Energy Appliances data set.
\begin{table}[!h]
    \centering
    \caption{Data Variables in the Energy Appliances Data Set}
    \begin{tabular}{l l c}
        \toprule
        \textbf{Data Variables} & \textbf{Units} & \textbf{Number of Features} \\
        \midrule
        Appliances energy consumption & Wh & 1 \\
        Light energy consumption & Wh & 2 \\
        $T_i$: Temperature in room $i$ & °C & 3-21 \\
        $RH_i$: Humidity in room $i$ & \% & 4-20 \\
        Temperature outside & °C & 21 \\
        Pressure  & mm Hg & 22 \\
        Humidity outside & \% & 23 \\
        Windspeed & m/s & 24 \\
        Visibility & km & 25 \\
        Dew point & °C & 26 \\
        Random Variable 1 (RV\_1) & - & 27 \\
        Random Variable 2 (RV\_2) & - & 28 \\
        Number of seconds from midnight  & s & 29 \\
        Week status (weekend or weekday) & - & 30 \\
        \bottomrule
    \end{tabular}
    \label{tab:energy_appliances}
\end{table}

\section{Major tasks and exercises will be done}
\subsection{Major tasks:}
\begin{itemize}
    \item Data cleaning and preprocessing, including missing value imputation,
    detecting and handling outliers, data transformation, etc.

    \item Exploratory data analysis, including data visualization, correlation analysis.

    \item Model building and comparison, including using different improvement techniques to
    improve the model performance, like feature selection, hyper-parameter\(\lambda\) tuning, etc.

    \item Model evaluation, including using different metrics to evaluate the model performance.
        Test the selected model on the test set and assess its performance.
\end{itemize}

\subsection{Corresponding exercises:}
\begin{itemize}
    \item Residual analysis.
    \item Multiple linear regression.
    \item Creating and using dummy variables.
    \item Variables selection.
    \item Perform statistics inference on a coefficient.
    \item *We might try to change the assumption about the residual in MLE for estimating 
    the coefficients in regression, which we had tried a little bit in the assignment 2 last week.
    But it depends on the time and the progress of the project.

    \href{https://github.com/Gufeng-2002/5320_Assignment_2/blob/main/LikelihoodEstimator.ipynb}{Click here to see the previous code of trying it.}
\end{itemize}

\section{Attestation}
We attest that \textbf{no team member} has analyzed the data in any relevant way previously.
We will also not plagiarize ideas or solutions from others to complete the project, even though other people 
might have done similar projects on top of the same data set.

\section{Temporary Schedule for the Project}
\begin{table}[h]
    \centering
    \begin{tabular}{|l|p{5cm}|l|l|}
        \hline
        \textbf{Tasks} & \textbf{Content} & \textbf{Name} & \textbf{Due Date} \\
        \hline
        Draft abstract & Summary of our report, covering the general work and conclusions & Yuzhuo Ye & mat 20th \\
        \hline
        Introduction & Why doing the research and how & Yuzhuo Ye & Mar 24th \\
        \hline
        Data preparation & Preparing, cleaning, transforming, and description & Anti Li & Mar 24th \\
        \hline
        Fitting model and doing tests & Searching a good model and making sure models pass statistical tests & Anti Li, Feng Gu & Mar 25th \\
        \hline
        Results & Explaining results for practical meanings, thinking drawbacks and possible improvements & Yuzhuo Ye, Feng Gu & Mar 27th \\
        \hline
        Discussion/Conclusion & Potential future work derived from our work and the following work that could be done to improve it. The meaning of our report in the real world. & Anti Li, Yuzhuo Ye & Mar 28th \\
        \hline
        Checking and refining & Checking core contents in the report, making possible improvements and organizing them in a clearer way. & Feng Gu & Mar 31th \\
        \hline
    \end{tabular}
    \caption{Overview of Tasks and Responsibilities}
\end{table}

\end{document}
